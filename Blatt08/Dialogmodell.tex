\documentclass{article}
\usepackage{tabularx}
\usepackage{booktabs}
\usepackage{multirow}
\usepackage[sfdefault]{roboto}
\begin{document}

\begin{tabularx}{\linewidth}{XX}
\toprule
{\Large \textbf{Zustände und Transitionen}} & {\Large \textbf{Konkrete Entsprechung im Movie Manager}} \\ \midrule

Z: Movie Master View & Sicht Movie Master View wird angezeigt \\
T: select "Search" & Tap auf den "Search" button \\
Z: Search Master View & Sicht Search Master View wird angezeigt \\
T: select "search only in Performers" & Tap auf den "Search only in Performers" Textlink \\
Z: Performer Search View & Sicht Performer Search View mit freiem Suchfeld wird angezeigt\\
T: search for a performer & Einen Suchstring in das Suchfeld eingeben \\
Z: Performer Search View & Sicht auf die Performer Search View mit Null oder mehreren Performern angezeigt\\
T: select \textless Performer\textgreater & Tap auf einen Performer in der Liste\\
Z: Performer Detail View & Sicht Performer Detail View wird angezeigt \\
\bottomrule

\end{tabularx}
In diesem Beispiel wird jeder Zustandsübergang von dem "H\textsuperscript{*}" beeinflusst.\\
Wenn immer ein Zustand existiert der Irgendwelche Daten halten kann wird er von dem "H\textsuperscript{*}" beeinflusst.\\
Angenommen das "H\textsuperscript{*}" wäre nicht da, dann würde wenn wir in der Performer Detail View zurück
gehen würden zu der Performer Search View in der Performer Search View die bereits gefundenen Performer von vorhin
nicht mehr da sein.\\
"H\textsuperscript{*}" bedeutet also, dass der Zustand sobald er wieder aufgerufen wird, seinen letzten Zustand wieder annimmt.
\end{document}